\documentclass[12pt,a4paper]{amsart}
% ukazi za delo s slovenscino -- izberi kodiranje, ki ti ustreza
\usepackage[slovene]{babel}
%\usepackage[cp1250]{inputenc}
%\usepackage[T1]{fontenc}
\usepackage[utf8]{inputenc}
\usepackage{amsmath,amssymb,amsfonts}
\usepackage{url}
%\usepackage[normalem]{ulem}
\usepackage[dvipsnames,usenames]{color}

% ne spreminjaj podatkov, ki vplivajo na obliko strani
\textwidth 15cm
\textheight 24cm
\oddsidemargin.5cm
\evensidemargin.5cm
\topmargin-5mm
\addtolength{\footskip}{10pt}
\pagestyle{plain}
\overfullrule=15pt % oznaci predlogo vrstico


% ukazi za matematicna okolja
\theoremstyle{definition} % tekst napisan pokoncno
\newtheorem{definicija}{Definicija}[section]
\newtheorem{primer}[definicija]{Primer}
\newtheorem{opomba}[definicija]{Opomba}

\renewcommand\endprimer{\hfill$\diamondsuit$}


\theoremstyle{plain} % tekst napisan posevno
\newtheorem{lema}[definicija]{Lema}
\newtheorem{izrek}[definicija]{Izrek}
\newtheorem{trditev}[definicija]{Trditev}
\newtheorem{posledica}[definicija]{Posledica}


% za stevilske mnozice uporabi naslednje simbole
\newcommand{\R}{\mathbb R}
\newcommand{\N}{\mathbb N}
\newcommand{\Z}{\mathbb Z}
\newcommand{\C}{\mathbb C}
\newcommand{\Q}{\mathbb Q}


% ukaz za slovarsko geslo
\newlength{\odstavek}
\setlength{\odstavek}{\parindent}
\newcommand{\geslo}[2]{\noindent\textbf{#1}\hspace*{3mm}\hangindent=\parindent\hangafter=1 #2}


% naslednje ukaze ustrezno popravi
\newcommand{\program}{Finančna matematika} % ime studijskega programa
\newcommand{\imeavtorja}{Katarina Brilej, Sara Kovačič} % ime avtorja
\newcommand{\imementorja}{prof.~dr. Riste Škrekovski} % akademski naziv in ime mentorja
\newcommand{\naslovdela}{Uporaba metahevristike GRASP na problemu potujočega trgovca}
\newcommand{\letnica}{2018} %letnica 


% vstavi svoje definicije ...




\begin{document}

% od tod do povzetka ne spreminjaj nicesar
\thispagestyle{empty}
\noindent{\large
UNIVERZA V LJUBLJANI\\[1mm]
FAKULTETA ZA MATEMATIKO IN FIZIKO\\[5mm]
\program\ -- 1.~stopnja}
\vfill

\begin{center}{\large
\imeavtorja\\[2mm]
{\bf \naslovdela}\\[10mm]
Projekt OR pri predmetu Finančni praktikum\\[1cm]
Mentor: \imementorja}
\end{center}
\vfill

\noindent{\large
Ljubljana, \letnica}
\pagebreak

\thispagestyle{empty}
\tableofcontents
\pagebreak



% tu se zacne besedilo seminarja
\section{Uvod}

\section{Problem potujočega trgovca}

Problem trgovskega potnika (”travelling salesman problem”/TSP) se glasi:
\begin{itemize}
\item {\bf Formulacija v vsakdanjem jeziku:} danih je $n$ mest in razdalja med poljubnim parom mest (od mesta do mesta lahko potujemo po zgolj eni poti). Najdi najkrajšo (najcenejšo) pot, ki se začne in konča v istem mestu ter obišče vsako mesto natanko enkrat.
\item{\bf Formulacija v matematičnem jeziku}: v (neusmerjenem enostavnem) polnem grafu $K_n$ z uteženimi povezavami (pozitivne vrednosti) najdi najkrajši cikel, ki vsebuje vsa vozlišča. Ciklom, ki vsebujejo vsa vozlišča grafa, pravimo Hamiltonovi cikli.

\end{itemize}

\subsection{Celoštevilski linearni program}







%\geslo{}{}
%
%\geslo{}{}
%


% seznam uporabljene literature
\begin{thebibliography}{99}

%\bibitem{}

\end{thebibliography}

\end{document}

