\documentclass[12pt,a4paper]{amsart}
% ukazi za delo s slovenscino -- izberi kodiranje, ki ti ustreza
\usepackage[slovene]{babel}
%\usepackage[cp1250]{inputenc}
%\usepackage[T1]{fontenc}
\usepackage[utf8]{inputenc}
\usepackage{amsmath,amssymb,amsfonts}
\usepackage{url}
%\usepackage[normalem]{ulem}
\usepackage[dvipsnames,usenames]{color}
\usepackage{graphicx}
\usepackage{float} % za fiksiranje slik na mestih (uporabi [H])
\usepackage[table,xcdraw]{xcolor}

\usepackage{listings}
\usepackage{color}
 
\definecolor{codegreen}{rgb}{0,0.6,0}
\definecolor{codegray}{rgb}{0.5,0.5,0.5}
\definecolor{codepurple}{rgb}{0.58,0,0.82}
\definecolor{backcolour}{rgb}{0.95,0.95,0.92}

\lstdefinestyle{mystyle}{
    backgroundcolor=\color{backcolour},   
    commentstyle=\color{codegreen},
    keywordstyle=\color{magenta},
    numberstyle=\tiny\color{codegray},
    stringstyle=\color{codepurple},
    basicstyle=\footnotesize,
    breakatwhitespace=false,         
    breaklines=true,                 
    captionpos=b,                    
    keepspaces=true,                 
    numbers=left,                    
    numbersep=5pt,                  
    showspaces=false,                
    showstringspaces=false,
    showtabs=false,                  
    tabsize=2
}
 
\lstset{style=mystyle}

% ne spreminjaj podatkov, ki vplivajo na obliko strani
\textwidth 15cm
\textheight 24cm
\oddsidemargin.5cm
\evensidemargin.5cm
\topmargin-5mm
\addtolength{\footskip}{10pt}
\pagestyle{plain}
\overfullrule=15pt % oznaci predlogo vrstico


% ukazi za matematicna okolja
\theoremstyle{definition} % tekst napisan pokoncno
\newtheorem{definicija}{Definicija}[section]
\newtheorem{primer}[definicija]{Primer}
\newtheorem{opomba}[definicija]{Opomba}

\renewcommand\endprimer{\hfill$\diamondsuit$}


\theoremstyle{plain} % tekst napisan posevno
\newtheorem{lema}[definicija]{Lema}
\newtheorem{izrek}[definicija]{Izrek}
\newtheorem{trditev}[definicija]{Trditev}
\newtheorem{posledica}[definicija]{Posledica}


% za stevilske mnozice uporabi naslednje simbole
\newcommand{\R}{\mathbb R}
\newcommand{\N}{\mathbb N}
\newcommand{\Z}{\mathbb Z}
\newcommand{\C}{\mathbb C}
\newcommand{\Q}{\mathbb Q}


% ukaz za slovarsko geslo
\newlength{\odstavek}
\setlength{\odstavek}{\parindent}
\newcommand{\geslo}[2]{\noindent\textbf{#1}\hspace*{3mm}\hangindent=\parindent\hangafter=1 #2}


% naslednje ukaze ustrezno popravi
\newcommand{\program}{Finančna matematika} % ime studijskega programa
\newcommand{\imeavtorja}{Katarina Brilej, Sara Kovačič} % ime avtorja
\newcommand{\imementorja}{prof.~dr. Riste Škrekovski} % akademski naziv in ime mentorja
\newcommand{\naslovdela}{Uporaba metahevristike GRASP na problemu potujočega trgovca}
\newcommand{\letnica}{2019} %letnica 


% vstavi svoje definicije ...




\begin{document}

% od tod do povzetka ne spreminjaj nicesar
\thispagestyle{empty}
\noindent{\large
UNIVERZA V LJUBLJANI\\[1mm]
FAKULTETA ZA MATEMATIKO IN FIZIKO\\[5mm]
\program\ -- 1.~stopnja}
\vfill

\begin{center}{\large
\imeavtorja\\[2mm]
{\bf \naslovdela}\\[10mm]
Projekt OR pri predmetu Finančni praktikum\\[1cm]
Mentor: \imementorja}
\end{center}
\vfill

\noindent{\large
Ljubljana, \letnica}
\pagebreak

\thispagestyle{empty}
\tableofcontents
\listoffigures
\listoftables

\pagebreak


% tu se zacne besedilo seminarja
\section{Uvod}

Metahevristika je algoritemski način reševanja kombinatoričnega optimizacijskega problema, pri katerem na začetku izberemo množico kandidatov za rešitev in jo iterativno izboljšujemo (glede na neko vnaprej izbrano funkcijo zaželenosti) ter po dovolj korakih vrnemo najboljši element iz te množice. Metahevristike torej vrnejo približne rešitve, a veliko hitreje kot eksaktni postopki. V projektu bomo na problem potujočega trgovca implementirali metahevristiko GRASP (\textit{greedy randomized adap\-tive search procedure}). Problem potujočega trgovca bomo rešili tudi kot celoštevilski linearni program in primerjali rešitve. Generirali bomo nekaj zanimivih grafov in na njih preizkusili algoritem. Rezultate bomo primerjail tudi z rezultati iz spleta in rezultati skupine 7, ki bo na problem potujočega trgovca implementirala genetski algoritem. 


\section{Problem potujočega trgovca} 

Problem potujočega trgovca (”travelling salesman problem”/TSP) se glasi:


\begin{itemize}
\item {\bf Formulacija v vsakdanjem jeziku:} danih je $n$ mest in razdalja med poljubnim parom mest (od mesta do mesta lahko potujemo po zgolj eni poti). Najdi najkrajšo (najcenejšo) pot, ki se začne in konča v istem mestu ter obišče vsako mesto natanko enkrat.
\item{\bf Formulacija v matematičnem jeziku}: v (neusmerjenem enostavnem) polnem grafu $K_n$ z uteženimi povezavami (pozitivne vrednosti) najdi naj\-krajši cikel, ki vsebuje vsa vozlišča. Ciklom, ki vsebujejo vsa vozlišča grafa, pravimo Hamiltonovi cikli.

\end{itemize}


\section{Grasp} 

GRASP (\textit{greedy randomized adaptive search procedure}) je metahevristika, ki sestoji iz dveh faz: \textit{greedy randomized construction} in \textit{local search}.
V prvi fazi na pameten način (odvisno od problema) izberemo izmed vseh možnih rešitev CL (candidate list) množico začetnih približkov RCL (restricted candidates list). 
To storimo deloma deterministično in deloma stohastično, da zagotovimo, da so začetni približki obe-tavni, a dovolj razpršeni po celotni množici CL, da bo druga faza pregledala čimvečji del CL. 
V drugi fazi za vsako izmed teh rešitev $ s \in  RCL $ pregledamo elemente $s' \in CL$ v njeni okolici (kaj je okolica je od problema in načina reševanja odvisno). Če najdemo boljšo rešitev $s'$, 
jo dodamo v RCL ter $s$ odstranimo. To ponavljamo dokler zaustavitveni pogoj (npr. št. iteracij, zahtevana natančnost) ni izpolnjen.

\subsection{Greedy randomized construction} 

Kot smo "ze omenili je GRASP sestavljen iz dveh faz. Najprej bomo predstavili prvo fazo oz. \textit{greedy randomized construction}.
Tu parameter alpha predstavlja mo"c mno"zice za"cetnih pribli"zkov (RCL), v na"sem primeru torej dol"zino seznama RCL. Algoritmu podamo tudi slovar razdalj (cen povezav) med mesti, n pa predstavlja dol"zino slovarja oz. "stevilo mest. Slovar je sestavljen iz elementov, ki imajo za klju"c povezavo oblike ($x_{i}$, $x_{j}$), za vrednost pa razdaljo/ceno, ki ji pripada.
Vsak za"cetni pribli"zek $t = (l,1,v_{2}, \dots,v_{n})$ $\in$ RCL konstruiramo tako, da dolo"cimo $ v_{1} := 1$ nato iterativno za $ i = 2, \dots,n$ za $v_{i}$ izberemo naklju"cno med p \% najbli"zjih vozli"s"c do $v_{i-1}$, ki "se niso v $t$. Na koncu "se izra"cunamo dol"zino poti s prej definirano funkcijo \textit{dolzina\_poti} in jo postavimo na ni"cto mesto. Take cikle $t$ konstruiramo toliko časa, da RCL napolnimo.   


\begin{lstlisting}[language=Python]
def greedy_construction(slovar,n,alpha):
    RCL = [0] * alpha
    if n > 5: # p mora biti vsaj 1
        p = n // 5 
    else:
        p = 1
    for j in range(0, alpha):
        t = [0] * (n+1)
        t[1] = 1
        mesta = [h for h in range(2,n+1)] 
        for i in range(2,n+1):
            povezave = [(t[i-1],m) for m in mesta] 
            slovar_povezav = { key:value for key, value in 
						slovar.items() if key in povezave }
            urejene_povezave = sorted(slovar_povezav, 
						key=slovar_povezav.__getitem__)
            (_,vi) = random.choice(urejene_povezave[:p]) 
            t[i] = vi
            mesta.remove(vi) 
        t[0] = dolzina_poti(slovar,n,t) 
        RCL[j] = t 
    return RCL 
\end{lstlisting}

\subsection{Local search} 

Drugo fazo algortima predstavlja \textit{local search}. Obravnavali bomo dve metodi: 2-opt in 3-opt. Funkcija \textit{local\_search} zato poleg matrike g, parametra alpha in "stevila iteracij sprejme "se parameter $metoda$. g je podana kot inciden"cna matrika razdalj med mesti/cen povezav velikosti $n \times n$, je simetri"cna ter z ni"clami na diagonali. Funkcija $slovar_razdalj$ matriko g pretvori v slovar kot je opisan v prvi fazi. 
V vrstici 2 definiramo RCL, ki predstavlja za"cetni seznam pribli"zkov. Te nato uredimo po dol"zini, primerjamo jih po prvem elementu, ki predstavlja dol"zino poti. Nato naklju"cno izberemo $t \in$ RCL, toda z linerano padajo"co verjetnostjo. Zato v vrstici 8 definiramo ute"zi. Najverjetneje izberemo cikel na vrhu RCL , torej najkraj"si. Ko imamo izbran $t$, se na podlagi parametra $metoda$ odlo"cimo za 2-opt ali 3-opt. Obe metodi poizku"sata pribli"zek izbolj"sati, "ce jima uspe, vrneta $novi\_t$. Neodvisno od metode nato v primeru, da smo pribli"zek uspeli izbolj"sati, v RCL dodamo izbolj"san pribli"zek in starega odstranimo. Kasneje si bomo ogledali "se kako posamezna metoda deluje. Ponavljamo tolikokrat kot je predpisano, to nam dolo"ca parameter $iter$. 

\begin{lstlisting}[language=Python]
def local_search(g,alpha,iter,metoda):
    n = len(g)
    slovar = slovar_razdalj(g) 
    RCL = greedy_construction(slovar,n,alpha)
    RCL.sort(key=lambda x: x[0]) 
    stevec = 0
    while stevec < iter: 
        utezi = [i * 2/((alpha+1)*alpha) for i in range(alpha,0,-1)]
        indeks = np.random.choice(len(RCL), size = 1, p = utezi) 
        t = RCL[indeks[0]]
        if metoda == "dva_opt":
            novi_t = dva_opt(n,t,slovar)
        elif metoda == "tri_opt":
            novi_t = tri_opt(n,t,slovar)
        if novi_t:
            RCL.append(novi_t) 
            RCL.remove(t) 
        stevec += 1
        RCL.sort(key=lambda x: x[0]) 
    return RCL[0] 
\end{lstlisting}

Sedaj si oglejmo kako delujeta posamezni metodi. Za"cnimo s preprostej"so, 2-opt. 
Ko naklju"cno izberemo cikel $t$, ga "zelimo izbolj"sati. Kraj"si cikel iščemo v okolici, ki je definirana kot množica vseh ciklov t' iz CL, 
ki jih dobimo iz t tako, da mu zamenjamo dve vozišči. Namesto, da bi shranjevali vse mo"zne  t' in na koncu preverili, "ce je najkraj"si kraj"si od trenutnega t, je bolj u"cinkovito, "ce sproti preverjamo, "ce posamezna menjava prinese izbolj"savo. Na za"cetku zato definiramo spremenljivko $razlika$, ta meri, "ce je dana menjava bolj"sa ("ce smo uspeli skraj"sati pot).
 Nato moramo v zanki lo"citi dva primera, saj v primeru, da je $j = n$, razdremo povezavo s prvim elementom. Nato izra"cunamo $change$, "ce je ta manj"si od trenutne razlike, jo posodobimo, saj smo dobili kraj"si cikel. Shranimo si tudi optimalni $i$ in $j$, da bomo na koncu vedeli s katero menjavo smo dosegli najkraj"si cikel. "Ce smo uspeli izbolj"sati t, bo razlika negativna in dobili bomo $novi\_t$. Tega konstruiramo tako, da na starem $t$ izvedemo menjavo dolo"ceno z optimalnima $i$ in $j$. Namesto, da znova ra"cunamo celotno dol"zino cikla, samo pri"stejemo razliko, ki je nagativna. 

\begin{lstlisting}[language=Python]
def dva_opt(n, t, slovar):
    razlika = 0
    for i in range(2,n):
        for j in range(i+1,n+1):
            if j != n: 
                change = slovar[(t[i-1],t[j])] + slovar[(t[i],t[j+1])] 
								- slovar[(t[i-1],t[i])] - slovar[(t[j],t[j+1])]
            else: 
                change = slovar[(t[i-1],t[j])] + slovar[(t[i],t[1])] - 
								slovar[(t[i-1],t[i])] - slovar[(t[j],t[1])]
            if change < razlika:
                razlika = change 
                opt_i =  i 
                opt_j = j 
    if razlika < 0: 
        novi_t = [t[m] for m in range(0,n+1)] 
        novi_t[opt_i:opt_j+1] = novi_t[opt_i:opt_j+1][::-1] 
        novi_t[0] = t[0] + razlika  
        return novi_t
    else:
        return None 
\end{lstlisting}

3-opt je nekoliko bolj dodelana razli"cica \textit{local searcha}, zamenjamo namre"c kar tri vozli"s"ca. 3-opt metoda nam omogo"ca pobeg v primeru, ko bi se z 2-opt metodo ujeli v lokalnem minimumu in ne bi mogli ven. Tako kot 2-opt funkcija sprejme velikost matrike, 
cikel $t$ in slovar povezav. Zaradi preglednosti na za"cetku definiramo spremenljivke X1, X2, Y1, Y2, Z1, Z2. Sedaj imamo 7 mo"znih menjav. Tri od teh so 2-opt (torej "ceprav razdremo tri povezave, nato eno spet pove"zemo nazaj), "stiri pa 3-opt (vse tri ostanejo nepovezane). Za vsako od mo"znih menjav nato izra"cunamo razliko. Ker pri 3-opt menjavah odstranimo enake povezave, lahko to shranimo pod spremenljivko $odstej$. Ko izra"cunamo vse razlike, poi"s"cemo minimum.
S tem ko si zabele"zimo, pri katerih indeksih je bila minimalna razlika dose"zena, bomo kasneje vedeli katero menjavo izvesti, saj so te urejene po vrsti v seznamu spremembe. "Ce je $change$ manj"sa od trenutne razlike, pomeni, da lahko $t$ izbolj"samo. Posodobimo $razliko$, $indeks$ in si zabele"zimo optimalne $i$,$j$ in $k$, ki jih bomo potem potrebovali, da izvedemo ustrezno menjavo. Torej "ce je na koncu razlika manj"sa kot 0, lahko zgradimo $novi\_t$. Ta je odvisen od vrste menjave, ki jo moramo izvr"siti, ta pa je dolo"cena z $indeksom$ in optimalnimi $i$,$j$ in $k$. Tako kot pri 2-opt tudi tukaj novo dol"zino cikla izra"cunamo tako, da pri"stejemo razliko. 
"Ce opi"semo se eno bolj zahtevno menjavo, recimo 3-opt menjavo change4. Prekinemo povezave (X1,X2), (Y1, Y2) in (Z1, Z2), zato te vrednosti od"stejemo. Na novo pa v tem primeru zgradimo povezave
(X1, Y1), (X2, Z1) in (Y2, Z2), zato te vrednosti pri"stejemo. Na ta na"cin izra"cunamo razliko v dol"zini cikla, ki jo ta menjava prinese, kako natan"cno pa to menjavo izvedemo bomo opisali pri funkciji \textit{menjava}. 
\begin{lstlisting}[language=Python]
def tri_opt(n, t, slovar):
    razlika = 0    
    for i in range(2,n-1):
        for j in range(i+1,n):
            for k in range(j+1,n+1):
                X1, X2, Y1, Y2, Z1, Z2 = t[i-1], t[i], t[j-1], t[j], 
								t[k-1], t[k]
                # 2 opt moves
                change1 = slovar[(X1,Z1)] + slovar[(X2,Z2)] -  
								slovar[(X1,X2)] - slovar[(Z1,Z2)]
                change2 = slovar[(Y1, Z1)] + slovar[(Y2, Z2)] -  
								slovar[(Y1, Y2)] - slovar[(Z1, Z2)] 
                change3 = slovar[(X1, Y1)] + slovar[(X2, Y2)] -  
								slovar[(X1, X2)] - slovar[(Y1, Y2)]
                # 3 opt moves
                odstej = slovar[(X1, X2)] + slovar[(Y1, Y2)] + 
								slovar[(Z1, Z2)]
                change4 = slovar[(X1, Y1)] + slovar[(X2, Z1)] + 
								slovar[(Y2, Z2)] -  odstej
                change5 = slovar[(X1, Z1)] + slovar[(Y2, X2)] + 
								slovar[(Y1, Z2)] -  odstej
                change6 = slovar[(X1, Y2)] + slovar[(Z1, Y1)] + 
								slovar[(X2, Z2)] -  odstej
                change7 = slovar[(X1, Y2)] + slovar[(Z1, X2)] + 
								slovar[(Y1, Z2)] -  odstej
                spremembe = [change1,change2,change3, change4, change5,
								change6,change7]
                change = min(spremembe)
                ind = np.argmin(spremembe) + 1 
                if change < razlika:
                    razlika = change
                    indeks = ind
                    opt_i =  i 
                    opt_j = j
                    opt_k = k
    if razlika < 0: 
        novi_t = menjava(indeks,n,t,opt_i,opt_j,opt_k) 
        novi_t[0] = t[0] + razlika 
        return novi_t
    else:
        return None
\end{lstlisting}

Funkcija \textit{menjava} sprejme indeks oz. katero mejavo "zelimo narediti, velikost matrike n, naklju"cno izbrani cikle $t$ $\in$ RCL ter optimalne $i$,$j$ in $k$ pri katerih je bila dose"zena minimalna razlika. 
"Ce opi"semo menjavo 4. Nove povezave (X1, Z1),(Y2, X2) in (Y1, Z1) dobimo tako da obrnemo del cikla med Y1 in X2 ter med Z1 in Y2.
Vse ostale menjave, prav tako pa izra"cuni razlik so opisani v datoteki GRASP.py.  


\begin{lstlisting}[language=Python]
def menjava(indeks,n,t,opt_i,opt_j,opt_k):   
    novi_t = [t[m] for m in range(0,n+1)]
    if indeks == 1:
        novi_t[opt_i:opt_k] = novi_t[opt_i:opt_k][::-1]
    if indeks == 2:
        novi_t[opt_j:opt_k] = novi_t[opt_j:opt_k][::-1]
    if indeks == 3:
        novi_t[opt_i:opt_j] = novi_t[opt_i:opt_j][::-1]      
    if indeks == 4:
        novi_t[opt_i:opt_j] = novi_t[opt_i:opt_j][::-1]
        novi_t[opt_j:opt_k] = novi_t[opt_j:opt_k][::-1]      
    if indeks == 5:
        tmp = novi_t[opt_j:opt_k][::-1] + novi_t[opt_i:opt_j]
        novi_t[opt_i:opt_k] = tmp        
    if indeks == 6:
        tmp = novi_t[opt_j:opt_k] + novi_t[opt_i:opt_j][::-1]
        novi_t[opt_i:opt_k] = tmp      
    if indeks == 7:
        tmp = novi_t[opt_j:opt_k] + novi_t[opt_i:opt_j]
        novi_t[opt_i:opt_k] = tmp
    return novi_t
\end{lstlisting}

\section{Celoštevilski linearni program} 

Problem potujočega trgovca lahko predstavimo kot \textit{celoštevilski linearni program}.
Označimo mesta s števili $1, \ldots, n$. Strošek (ali razdalja) potovanja iz mesta $i$ v mesto $j$ je $c_{i,j}$, ( $ 1\leq i, j\leq n$). Minimiziramo strošek potovanja. Definiramo:

$$ X_{i,j} := \left\{ \begin{array}{ll}
1 ~; & \textrm{potnik gre iz mesta $i$ v mesto $j$}, \\
0 ~; & \textrm{sicer},
\end{array} \right. $$

\hfill \break

$y_{i}$ \ldots katero po vrsti obiščemo mesto $i$

\begin{equation*}
\begin{aligned}
& {\text{min}}
& & \sum_{i=1}^{n} \sum_{j=1}^{n}  x_{i,j} \cdot c_{i,j} \\
& \text{p. p.}
& &\sum_{i=1}^{n} x_{i,j} = 1, \text{ za vsak $j$ }\\
&&&\sum_{j=1}^{n} x_{i,j} = 1, \text{ za vsak $i$ }\\
&&&  x_{i,j} \in \{0,1\}, \text{ za vsak $i$ in vsak $j$ }\\
&&& y_{i} \in \{1, \ldots, n\}; \text{za vsak $i$}\\
&&& y_{i} + 1 - n + n \cdot x_{i,j} \leq y_{j}; \text{za vsak $i$ in vsak $j>1$}
\end{aligned}
\end{equation*}
\\
Problem potujočega trgovca se da v Pythonu elegantno zapisati kot celoštevilski linearni program s pomočjo knjižnice PuLP. PuLP za reševanje problema izbere enega od vgrajenih algoritmov.
 V našem primeru je PuLP za reševanje izbral PULP\_CBC\_CMD. S pomočjo knjižnice PuLP smo napisali funckijo, ki sprejme matriko cen povezav, izpiše minimalno ceno potovanja in vrne urejen seznam obiska\-nih mest.
 V nadaljevanju je prikazan del funkcije $ tsp\_as\_ilp$, v katerem definiramo problem in ga rešimo. Na repozitoriju se nahaja celotna funckija, ki vrne urejen seznam obiskanih mest.

\begin{lstlisting}[language=Python]
from pulp import *
def tsp_as_ilp(g):
    razdalje = slovar_cen_ilp(g)   # slovar razdalj 
    mesta = [x+1 for x in range(len(g))]  # seznam mest (od 1 do n)
    prob = LpProblem("salesman", LpMinimize) # definiramo problem
    x = LpVariable.dicts('x', razdalje, 0,1, LpBinary) 
    cena = lpSum([x[(i,j)]*razdalje[(i,j)] for (i,j) in razdalje]) 
    prob += cena
    for k in mesta:  # pri pogojih:
        prob += lpSum([ x[(i,k)] for i in mesta if (i,k) in x]) == 1  
        prob += lpSum([ x[(k,i)] for i in mesta if (k,i) in x]) == 1  
    u = LpVariable.dicts('u', mesta, 0, len(mesta)-1, LpInteger) 
    N = len(mesta)
    for i in mesta:  
        for j in mesta:
            if i != j and (i != 1 and j!= 1) and (i,j) in x:
                prob += u[i] - u[j] <= (N)*(1-x[(i,j)]) - 1
    prob.solve() # PuLP sam izbere metodo
    print("Optimalna cena potovanja = ", value(prob.objective))   
\end{lstlisting}



\section{Primerjave}
Za razli"cne primerjave smo se osredoto"cili na pet grafov, vendar pa ne bomo povsod primerjali vseh. Poleg grafov skupine 7, ti so \textit{Ulysses22}, \textit{Berlin52} in \textit{KroA100} smo si izbrali "se \textit{Swiss42} in \textit{St70}. Vse smo dobili iz TSPLIB, kjer so zapisane tudi trenutne znane optimalne re"sitve, ki jih bomo tekom naslednjega poglavja tudi primerjali. Ker pa so podatki za TSP podani v razli"cnih oblikah, smo napisali "se tri razli"cne funkcije za uvoz. Za primerjavo GRASP in ILP pa smo sami generirali nekoliko manj"se matrike. Te so simetri"cne s celo"stevilskimi vrednostmi od 1 pa do maximalne vrednosti, ki jo dolo"cimo sami. 

\begin{lstlisting}[language=Python]
def TSP(N, max_pot, min_pot = 1):
    " funkcija vrne nakljucno matriko cen povezav, ki predstavlja problem potujocega trgovca"
    a = np.random.randint(min_pot, max_pot, size= (N,N))
    m = np.tril(a) + np.tril(a, -1).T
    for i in range(N):
        m[i][i] = 0
    return m
\end{lstlisting}

\subsection{Primerjava ILP in GRASP} ~\\
Z merjenjem časa reševanja določenih primerov različnih velikosti ugotovimo, da napisana funkcija $ilp$ porabi več časa za reševanje, kot metahevristika $GRASP$. Za potrebe primerjave, smo za parametre pri $GRASP$ vzeli alpha=3, št. iteracij=100 in metodo 3-opt.
V nadaljevanju bomo ugotovili, da na čas izvajanja local\_search-a vplivata tudi število iteracij in predvsem metoda, zato časovna primerjava funkcij $ tsp\_as\_ilp$ in $GRASP$ ni najbolj točna. Ker za majhne grafe $GRASP$ tudi pri 100 iteracijah vrne pravilno (optimalno) rešitev,
lahko vseeno sklepamo, da deluje hitreje kot $ tsp\_as\_ilp$. 


\begin{figure}[H]
\caption{Časovna primerjava ILP in GRASP}
\centering
\includegraphics[scale =0.8]{casilp}
\end{figure}


\subsection{Primerjava parametra alpha} ~\\

Za primerjavo parametra alpha sva se osredoto"cili na "stiri parametre in sicer 3, 5, 10 in 15. Parameter alpha nam dolo"ca velikost RCL, torej "stevilo za"cetnih pribli"zkov. "Ce je alpha enak 5, bo funkcija \textit{greedy\_construction} skonstruirala 5 za"cetnih pribli"zkov. Te bo nato \textit{local\_search} izbolj"seval. Za za"cetek smo primerjavo naredili za obe metodi na 100 iteracijah, naredili smo deset ponovitev algoritma. "Ce si najprej ogledamo najmanj"si graf \textit{Ulysses22}, lahko vidimo, da izbira parametra ne vpliva na rezultat, saj v vseh primerih algoritem vrne znan optimalni rezultat 1273. Pri malo ve"cjem grafu \textit{Swiss42} "ze opazimo, da se rezultat oddaljuje od optimuma, ve"cji kot je alpha. Vendar pa je pri 2-opt slab"sanje o"citno hitrej"se. Pri 3-opt celo kljub slab"sem rezultatu pri alpha = 10, pri alpha = 15 vrne optimalen rezultat. Optimalen rezultat za \textit{Berlin52} je 7544, mi pa smo dobili 7616, kar je precej blizu. Se pa tu rezultat "se hitreje poslab"sa z večanjem alpha, zopet pri 2-opt bolj kot pri 3-opt. Tudi pri \textit{St70} najbolj"si rezultat vrne 3-opt, spet pri najmanj"sem izbranem alpha. Za \textit{KroA100} so rezultati že močno oddaljeni od optimuma vse bolj kot se pove"cuje alpha. 
"Ce povzamemo. Opazimo, da razen v primeru zelo majnih grafov, z uporabo manjšega alpha (npr. 3), dobimo boljši rezultat. Ve"canje alpha pa bolj vpliva na metodo 2-opt kot 3-opt, saj se ta za"cne hitreje odmikati od optimuma. Prav tako ima spreminjanje parametra ve"cji vpliv pri ve"cjih grafih. Opazimo tudi, da pri enakem "stevilu iteracij metoda opt-3 na"celoma vrne bolj"si rezultat. To je smiselno, saj vsaki"c pregleda ve"cjo okolico cikla, ki ga "zelim oizbolj



\begin{table}[H]
\caption{Primerjava parametra alpha, iter = 100, ponovitve = 10}
\begin{tabular}{lllllll}
\rowcolor[HTML]{FFCCC9} 
TSP       &      & 3 & 5 & 10 & 15 &opt \\ \hline
Ulysses22 &      &   &   &    &  &7013  \\
          & 2-opt & 7013  &  7013 & 7013   & 7013   &\\
          & 3-opt &   7013&  7013 &  7013  &  7013 & \\
Swiss42   &      &   &   &    &   &1273 \\
	& 2-opt &   1273&  1273 &   1420 &  1592 & \\
          & 3-opt &  1273 &  1273 & 1316   &  1273 & \\
Berlin52 &      &   &   &    &   &7542 \\
	 & 2-opt &  7716 & 8130 &   8629 &  10464 & \\
          & 3-opt &  7616 &  7684 &  7735  & 8705  & \\
St70      &      &   &   &    &  & 675 \\
	& 2-opt &  714 &  888 &   1211 &  1386 & \\
          & 3-opt &  684 &  754 &   878 &   941& \\
KroA100   &      &   &   &    &  &21282  \\
	& 2-opt &  37162 &  54119 &   74951 &   74462& \\
          & 3-opt &  22917 &  26938 &  41953  & 46026  &
\end{tabular}
\end{table}

Za naslednjo primerjavo smo se odlo"cili primerjati "se razli"cne parametre alpha pri "stevilu iteracij = 1000. Zdaj smo primerjali samo metodo 2-opt, dodali pa smo "se alpha = 30. Zopet pa smo naredili 10 ponovitev. 
Opazimo, da zdaj tudi za primer \textit{Swiss42} metoda vrne optimalno vrednost, neodvisno od alpha. Pri grafu \textit{Berlin52} se vrednosti za vse prej"snje alpha mo"cno izbolj"sajo, nekolik"sno odstopanje lahko zasledimo le pri alpha = 30. Tudi pri primerih \textit{St70} in  \textit{KroA100} se vrednosti precej bolje pribli"zajo optimalni, tudi oddaljevanje od optimuma v odvisnosti z ve"canjem alpha je po"casnej"se kot pri iter = 100. Zopet lahko najmanj"se vrednosti v povpre"cju opazimo pri manj"sih alpha, vendar pa se zdaj vpliv izbranih parametrov alpha zmanj"sa. Sklepamo lahko torej, da si lahko pri ve"cjem "stevilu iteracij lahko "privo"scimo" ve"cji alpha in bo algoritem "se vedno vra"cal optimalno re"sitev. Prav tako so re"sitve pri danem "stevilu iteracij bolj"se za manj"se grafe. 

\begin{table}[H]
\caption{Primerjava parametra alpha, iter = 1000, ponovitve = 10}
\begin{tabular}{llllllll}
\rowcolor[HTML]{FFCCC9} 
TSP      &       & 3     & 5     & 10    & 15   & 30   & opt   \\
Swiss42 &       &       &       &       &      &      & 1273  \\
         & opt-2 & 1273  & 1273  & 1273  & 1273 & 1273 &       \\
Berlin52  &       &       &       &       &      &      & 7542  \\
         & opt-2 & 7544  & 7544  & 7544  & 7544 & 7560 &       \\
St70     &       &       &       &       &      &      & 675   \\
         & opt-2 & 678   & 685   & 678   & 682  & 685  &       \\
KroA100  &       &       &       &       &      &      & 21282 \\
         & opt-2 & 21381 & 21709 & 21642 &  22009    &  24275    &      
\end{tabular}
\end{table}

\subsection{Primerjava glede na število iteracij} ~\\

Pri določanju parametrov algoritma ima pomembno vlogo tudi število iteracij. Odvisnost vrednosti rešitve od števila iteracij smo preizkusili na primeru St70, velikosti 70. Iz prejšnjega podpoglavja ugotovimo, da je za ta primer najboljši parameter alpha = 3. Zaradi velike časovne zahtevnosti metode 3-opt, smo za ta primer naredili le 200 iteracij. Ugotovimo, da se vrednost rešitve izboljšuje z večanjem iteracij. Metoda 3-opt je pri preizkušanju dosegla dober rezultat (684) že pri 100 iteracijah. Podobno rešitev pri metodi 2-opt dobimo šele pri 600 iteracijah, najboljšo pa pri 1000.

\begin{figure}[H]
\caption{Odvisnost rešitve od števila iteracij}
\centering
\includegraphics[scale =0.8]{resitev_iteracije1}
\end{figure}

Zaradi manjše časovne zahtevnosti metode 2-opt, lahko pri izvajanju GRASP-a uporabimo več iteracij kot pri metodi 3-opt. Metoda 3-opt uporabi $\mathcal{O}(n^3)$ operacij, 2-opt pa ima kvadratično časovno zahtevnost oz. $\mathcal{O}(n^2)$.

\begin{figure}[H]
\caption{Odvisnost časa izvajanja od števila iteracij}
\centering
\includegraphics[scale =0.8]{cas_izvajanja}
\end{figure}

\subsection{Primerjava s skupino 7} ~\\

\begin{table}[H]
\begin{tabular}{lll}
\rowcolor[HTML]{FFCCC9} 
          & GRASP & GA \\
Ulysses22 & 7013  & 7112               \\
Berlin52  & 7544  & 8737               \\
KroA100   & 21381 & 36408             
\end{tabular}
\end{table}

Skupina 7 je na problem potujočega trgovca implementirala Genetski Algoritem (GA). V njihovem poročilu smo poiskali najboljše rezultate posameznega grafa in jih zapisali v tabelo. Iz tabele razberemo, da se je algoritem GRASP odrezal bolje od genetskega algoritma.
Tudi iz vira [6] razberemo, da je genetski algoritem sicer natančnejši, vendar se na primerih iz prakse pokaže, da natančnost ni tako pomembna, in zato GRASP deluje bolje od GA.

\subsection{Vizualizacija} ~ \\
Za boljšo predstavo smo napisali še dve funkciji, ki narišeta podano omrežje in rešitev na njem.
Če imamo podane koordinate, funkcija $narisi$ nariše točke kot koordinate v 2D in jih poveže glede na dano rešitev.
\begin{figure}[H]
\caption{Najkraša pot primera Berlin52 dolga 7544}
\centering
\includegraphics[scale =0.5]{berlin_7544}
\end{figure}

\begin{figure}[H]
\caption{Najkrajša pot primera Ulysses22, dolga 7013}
\centering
\includegraphics[scale =0.5]{ulysses22_7013}
\end{figure}

\begin{figure}[H]
\caption{Najkrajša pot primera KroA100, dolga 21381}
\centering
\includegraphics[scale =0.5]{kroA100_21381}
\end{figure}

Če pa koordinat nimamo podanih, pa funckija $narisigraf(matrika)$ sprejme matriko uteži povezav in nariše usmerjen graf z utežmi na povezavah. Napisali smo še funkcijo, ki rešitev algoritma GRASP spremeni v matriko, tako da še lahko narišemo rešitev. 
Opomba: pri velikih grafih je risanje vseh povezav nesmiselno, zato te funkcije nismo izboljševali.
\begin{figure}[H]
\caption{Poln graf in primer rešitve za Swiss42}
\centering
\includegraphics[scale =0.5]{swiss_primerjava}
\end{figure}
\pagebreak

\section{Zaključek}

Problem potujočega trgovca je NP-težak problem v kombinatorični optimizaciji, ki je pomemben v operacijskih raziskavah in računalniških znanostih.  Problem je bil prvič oblikovan leta 1930 in je eden izmed najbolj intenzivno proučevanih problemov optimizacije. Na problem smo implementirali metahevristiko GRASP, ki sestoji iz dveh faz (Greedy randomized construction in Local search). V Local search-u smo obravnavali metodi 2-opt in 3-opt, kjer je glavna razlika to, da pri 2-opt zamenjamo dve vozlišči, pri 3-opt pa tri. Metoda 3-opt vsebuje tri for zanke, zato je njena časovna zahtevnost kar  $\mathcal{O}(n^3)$. Vsak 3-opt premik je bodisi enak 2-opt premiku ali je enak zaporedju dveh ali treh 2-opt potez. Čeprav je 3-opt bolj zapleten in počasnejši, se uporablja, ker je možno, da obstaja zaporedje 2-opt premikov, ki izboljšajo pot vendar se začne z 2-opt premikom, ki poveča dolžino poti, zaradi česar to zaporedje premikov ne bo izvedeno.  V Pythonu lahko problem potujočega trgovca zapišemo kot celoštevilski linearni program (ILP) s pomočjo knjižnice PULP, ki za reševanje problema izbere enega od vgrajenih algoritmov. Z merjenjem časa izvajanja funkcij, smo ugotovili, da GRASP (100 iteracij, metoda 3opt) deluje hitreje kot ILP.  Primerjave različnih parametrov metode GRASP smo izvedli na grafih \textit{Ulysses22}, \textit{Berllin52}, \textit{KroA100}, \textit{Swiss42} in \textit{St70}.  Najprej smo primerjali parameter alpha, ki določa koliko začetnih rešitev problema bo skonstruiral algoritem. Na rešitve manjših grafov parameter alpha ni vplival, pri večjih grafih pa je manjši alpha pomenil boljšo rešitev. Zato smo v nadaljevanju uporabljali vrednost alpha = 3. Prav tako smo ugotovili, da lahko pri večjem "stevilu iteracij vzamemo ve"cji alpha da bo re"sitev "se vedno optimalna. Tako pri metodi 2-opt in 3-opt smo z ve"canjem alpha zasledili oddaljevanje od optimuma (vsaj pri ve"cjih grafih), oddaljevanje pa je bilo hitrej"se pri metodi 2-opt. 
Vrednost rešitve se izboljšuje z večanjem števila iteracij, vendar se s tem veča tudi časovna zahtevnost. Metoda 3-opt je na primeru \textit{St70} za 200 iteracij potrebovala že skoraj 200 sekund, vendar je že pri 100 iteracijah vrnila podoben rezultat kot 2-opt pri 1000 ali 2000. Metoda 2opt tudi za 2000 iteracij potrebuje manj kot 30 sekund. Rezultate algoritma GRASP smo na primerih \textit{Ulysses22}, \textit{Berllin52} in \textit{KroA100} primerjali tudi s skupino 7, ki je izvajala Genetski algoritem (GA). Ugotovili smo, da so se na"sim rezultatom najbolje približali pri najmanjšem grafu \textit{Ulysses22}, kjer GRASP zelo hitro vrne kar optimalno rešitev (7013). Pri grafu velikosti 100 (KroA100), pa je GRASP deloval veliko bolje, saj se je od znane optimalne rešitve (21282) razlikoval za 99, rešitev genetskega algoritma pa se je od te rešitve razlikovala za kar 15126. Ugotovimo, da je algoritem zaradi svoje enostavnosti na splošno zelo hiter in je zmožen producirati kar dobre rezultate v kratkem času. 


\pagebreak
% seznam uporabljene literature
\begin{thebibliography}{99}

\bibitem{Wikipedia TSP}
Travelling salesman problem
\\\texttt{http://en.wikipedia.org/wiki/Travelling\_salesman\_problem}

\bibitem{}
C. Blum, A. Roli, Metaheuristics in Combinatorial Optimization: Overview and Conceptual Comparison, online.
\\\texttt{https://www.iiia.csic.es/~christian.blum/downloads/blum\_roli\_2003.pdf}

\bibitem{}
S. Luke, Essentials of Metaheuristics: a set of undergraduate lecture notes, online.
\\\texttt{https://cs.gmu.edu/~sean/book/metaheuristics/Essentials.pdf}

\bibitem{}
TSP Basics
\\\texttt{https://tsp-basics.blogspot.com/2017/03/}

\bibitem{}
R. Škrekovski, Zbornik seminarjev iz hevristik [Elektronski vir]: izbrana poglavja iz optimizacijskih metod (2010-11) / Riste Škrekovski, Vida Vukašinovič - El. knjiga. - Ljubljana: samozal. R. Škrekovski, 2012. 
\\\texttt{https://www.fmf.uni-lj.si/~skreko/Gradiva/Zbornik\_Hevristike.pdf}

\bibitem{}
B. Bernai, S. Deleplanque, A. Quilliot, Routing on Dynamic Networks: GRASP versus Genetic.
\\\texttt{https://annals-csis.org/Volume\_2/pliks/52.pdff}

\end{thebibliography}

\end{document}

